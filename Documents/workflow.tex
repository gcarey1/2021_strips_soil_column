% Options for packages loaded elsewhere
\PassOptionsToPackage{unicode}{hyperref}
\PassOptionsToPackage{hyphens}{url}
%
\documentclass[
]{article}
\usepackage{amsmath,amssymb}
\usepackage{lmodern}
\usepackage{ifxetex,ifluatex}
\ifnum 0\ifxetex 1\fi\ifluatex 1\fi=0 % if pdftex
  \usepackage[T1]{fontenc}
  \usepackage[utf8]{inputenc}
  \usepackage{textcomp} % provide euro and other symbols
\else % if luatex or xetex
  \usepackage{unicode-math}
  \defaultfontfeatures{Scale=MatchLowercase}
  \defaultfontfeatures[\rmfamily]{Ligatures=TeX,Scale=1}
\fi
% Use upquote if available, for straight quotes in verbatim environments
\IfFileExists{upquote.sty}{\usepackage{upquote}}{}
\IfFileExists{microtype.sty}{% use microtype if available
  \usepackage[]{microtype}
  \UseMicrotypeSet[protrusion]{basicmath} % disable protrusion for tt fonts
}{}
\makeatletter
\@ifundefined{KOMAClassName}{% if non-KOMA class
  \IfFileExists{parskip.sty}{%
    \usepackage{parskip}
  }{% else
    \setlength{\parindent}{0pt}
    \setlength{\parskip}{6pt plus 2pt minus 1pt}}
}{% if KOMA class
  \KOMAoptions{parskip=half}}
\makeatother
\usepackage{xcolor}
\IfFileExists{xurl.sty}{\usepackage{xurl}}{} % add URL line breaks if available
\IfFileExists{bookmark.sty}{\usepackage{bookmark}}{\usepackage{hyperref}}
\hypersetup{
  hidelinks,
  pdfcreator={LaTeX via pandoc}}
\urlstyle{same} % disable monospaced font for URLs
\usepackage[margin=1in]{geometry}
\usepackage{graphicx}
\makeatletter
\def\maxwidth{\ifdim\Gin@nat@width>\linewidth\linewidth\else\Gin@nat@width\fi}
\def\maxheight{\ifdim\Gin@nat@height>\textheight\textheight\else\Gin@nat@height\fi}
\makeatother
% Scale images if necessary, so that they will not overflow the page
% margins by default, and it is still possible to overwrite the defaults
% using explicit options in \includegraphics[width, height, ...]{}
\setkeys{Gin}{width=\maxwidth,height=\maxheight,keepaspectratio}
% Set default figure placement to htbp
\makeatletter
\def\fps@figure{htbp}
\makeatother
\setlength{\emergencystretch}{3em} % prevent overfull lines
\providecommand{\tightlist}{%
  \setlength{\itemsep}{0pt}\setlength{\parskip}{0pt}}
\setcounter{secnumdepth}{-\maxdimen} % remove section numbering
\ifluatex
  \usepackage{selnolig}  % disable illegal ligatures
\fi

\author{}
\date{\vspace{-2.5em}}

\begin{document}

\hypertarget{recommended-collaboration-workflow-for-git}{%
\subsection{Recommended collaboration workflow for
Git}\label{recommended-collaboration-workflow-for-git}}

Generally, you want to only update the main/master branch via pull
requests from branches that have been merged with master beforehand.
This is so that

\begin{enumerate}
\def\labelenumi{\arabic{enumi}.}
\tightlist
\item
  Main/master branch is always in a state that ``works''
\item
  Any potential merge conflicts with main have been addressed in the
  branches before the branch is incorporated into main, making sure that
  adding new changes to the main branch won't break it.
\end{enumerate}

\hypertarget{workflow-example}{%
\subsubsection{Workflow example}\label{workflow-example}}

Let's say you're working on your own. You have the \texttt{main} branch
that has the current working site. An example workflow would be:

\begin{enumerate}
\def\labelenumi{\arabic{enumi}.}
\tightlist
\item
  \texttt{git\ checkout\ main} followed by \texttt{git\ pull} to make
  sure you have the most up-to-date version of \texttt{main}.

  \begin{itemize}
  \tightlist
  \item
    Ideally, there should be no merge conflicts or problems here because
    you've been following this workflow :)
  \end{itemize}
\item
  \texttt{git\ checkout\ -b\ your-new-branch} to create a new branch
  named \texttt{your-new-branch} and switch to it, followed by
  \texttt{git\ push\ -u\ origin\ your-new-branch} to tell the remote
  repo about your branch.
\item
  Make all your changes on the \texttt{your-new-branch} branch:
  \texttt{git\ add} and \texttt{commit} as you normally would. When
  you're ready to move all the changes to \texttt{main}, do a
  \texttt{git\ push} to update the remote
  \texttt{origin/your-new-branch}.
\item
  \texttt{git\ checkout\ main} followed by \texttt{git\ pull} to update
  \texttt{main} again.

  \begin{itemize}
  \tightlist
  \item
    If you're working on your own, there probably won't be many changes,
    but if you have collaborators, this is necessary to make sure you
    know about changes made by other people.
  \end{itemize}
\item
  \texttt{git\ checkout\ your-new-branch} to switch to the new branch
  you made
\item
  \texttt{git\ merge\ main} to add all the changes from \texttt{main}
  into \texttt{your-new-branch}.

  \begin{itemize}
  \tightlist
  \item
    If there are no merge conflicts, awesome!
  \item
    If there are, you'll have to reconcile the merges. You can either
    fix the merges in \texttt{vim}/your favorite IDE, or you can do use
    the \texttt{theirs/ours} merging strategies to accept all the
    changes from the \texttt{main}/\texttt{your-new-branch} branches,
    respectively. There are two ways you can do this:

    \begin{itemize}
    \tightlist
    \item
      Abort the merge via \texttt{git\ merge\ -\/-abort} to cancel the
      merge attempt, then re-do the merge with
      \texttt{git\ merge\ -s\ ours/theirs\ master} to take all the
      changes from \texttt{your-new-branch}/\texttt{main} respectively.
    \item
      If you're already in a conflicted state and you want to use one of
      the strategies, you can use
      \texttt{git\ checkout\ -\/-theirs/-\/-ours\ .} to take all the
      changes from \texttt{your-new-branch}/\texttt{main} respectively,
      followed by \texttt{git\ add\ .} and \texttt{git\ commit}.

      \begin{itemize}
      \tightlist
      \item
        More info:
        \url{https://stackoverflow.com/questions/10697463/resolve-git-merge-conflicts-in-favor-of-their-changes-during-a-pull/33569970\#33569970}
      \end{itemize}
    \end{itemize}
  \item
    After handling the merge, \texttt{git\ push} again to update
    \texttt{origin/your-new-branch}.
  \end{itemize}

  At this point you have successfully incorporated all the changes in
  \texttt{main} into \texttt{your-new-branch} and resolved any
  conflicts. Incorporating all the changes from \texttt{your-new-branch}
  into \texttt{main} should be painless now (assuming no one has made
  any changes between the last step and here).
\item
  To incorporate the changes from \texttt{your-new-branch} into main,
  there are two main options:

  \begin{enumerate}
  \def\labelenumii{\arabic{enumii}.}
  \tightlist
  \item
    To just merge all the changes without review, do
    \texttt{git\ checkout\ main} to checkout the main branch followed by
    \texttt{git\ merge\ your-new-branch}. This should just be a
    \texttt{fast-forward} merge since you handled all the conflicts in
    step 6. Now do \texttt{git\ push} to update \texttt{origin/main}.
    You're all done!
  \item
    If you want someone else to review the changes before merging the
    two branches together, you can open a pull request on the Github
    website:

    \begin{itemize}
    \tightlist
    \item
      Go to the repo website and click on ``Pull Requests''. Click on
      ``New Pull Request''.
    \item
      The base branch should be \texttt{main/master} and the
      \texttt{compare} branch should be \texttt{your-new-branch}. Write
      something about what the pull request does and submit the pull
      request.
    \item
      Theoretically, you should see some checks underneath that say
      something like ``This branch has no conflicts with the base
      branch'' (since we handled all the conflicts in step 6) and you
      should see the beautiful green ``Merge pull request button''. Then
      whoever you want to review the changes will be able to click that
      button and merge all the changes into the main branch.
    \item
      In order to get all the changes into your local, you'll need to do
      \texttt{git\ checkout\ main} followed by \texttt{git\ pull}.
      Again, this should be a \texttt{fast-forward} merge. See
      \href{https://github.com/pommevilla/friendly-dollop/pull/2}{here}
      for an example of someone following this workflow to submit
      changes into my repo.
    \end{itemize}
  \end{enumerate}
\end{enumerate}

\end{document}
